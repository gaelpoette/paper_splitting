\documentclass[a4paper,11pt]{report}
%draft sert � compiler sans charger les images, c'est beaucoup moins long
% charger des modules � utiliser ...
\usepackage[english]{babel}
\usepackage[latin1]{inputenc}
\usepackage{amsmath}
\usepackage{fullpage}
\usepackage{delarray}
\usepackage{afterpage}
\usepackage{graphicx}
\usepackage{makeidx}
\usepackage{xcolor}

%\usepackage[T1]{fontenc}
\usepackage{latexsym}
\usepackage{amsfonts}
\usepackage{amssymb}

% c'est moche mais c'est pratique pour rep�rer les liens morts etc...
%\usepackage{hyperref}
\usepackage{backref}
\usepackage{setspace}
\singlespacing

\bibliographystyle{plain}

\usepackage{listings}
\usepackage{calc}
\usepackage{epsfig}



% NOUVEAUX ENVIRONNEMENTS
%\newtheorem{theoreme}{Th�or�me}[chapter]
%\newtheorem{corollaire}{Corollaire}[chapter]
%\newtheorem{lemme}{Lemme}[chapter]
%\newtheorem{proposition}{Proposition}[chapter]
%\newtheorem{definition}{D�finition}[chapter]
%\newtheorem{propriete}{Propri�t�}[chapter]
%\newtheorem{hypotheses}{Hypoth�ses}[chapter]
\newenvironment{preuve}[1][Preuve]{\begin{trivlist}\item[\hskip \labelsep {\bfseries #1}]}{\hfill$\blacksquare$ \end{trivlist}}
\newenvironment{remarque}[1][Remarque]{\begin{trivlist}\item[\hskip \labelsep {\bfseries #1}]}{\hfill$\bullet$ \end{trivlist}}
\newenvironment{temporaire}[1][Commentaire temporaire]{\begin{trivlist}\item[\hskip \labelsep {\bfseries #1}]}{\hfill\checkmark \end{trivlist}}

\newcommand{\sousparagraphe}[1]{\textbf{\textit{#1}}}
\newcommand{\cacher}[1]{}
\newcommand{\mbf}[1]{{\mathversion{bold} #1}}
\newcommand{\Frac}[2]{\frac{\displaystyle #1}{\displaystyle #2}}
\newcommand{\Lim}[2]{\displaystyle \lim_{#1}{#2}}
\newcommand{\Int}{\displaystyle \int}

\newcommand{\INT}[4]{\Int_{#1}^{#2} #3 \: \textnormal{d}#4}
\newcommand{\Sum}{\displaystyle \sum}
\newcommand{\Prod}{\displaystyle \prod}
\newcommand{\SUM}[3]{\Sum_{#1}^{#2} #3}
\newcommand{\PD}[3]{\left.\Frac{\partial #1}{\partial #2}\right|_{#3}}
\newcommand{\TD}[2]{\Frac{\textnormal{d} #1}{\textnormal{d} #2}}
\newcommand{\dif}[1]{\textnormal{d} #1}
\newcommand{\pdt}{\partial_t}
\newcommand{\pdx}{\partial_x}
\newcommand{\pdy}{\partial_y}
\newcommand{\pdz}{\partial_z}
\newcommand{\pdX}{\partial_X}
\newcommand{\pdm}{\partial_m}
\newcommand{\pdtt}{\partial_{t^2}^2}
\newcommand{\pdxx}{\partial_{x^2}^2}
\newcommand{\pdxi}{\partial_{x_i}}
\newcommand{\pdxj}{\partial_{x_j}}
\newcommand{\pdxixi}{\partial_{x_i x_i}^2}
\newcommand{\pdxjxj}{\partial_{x_j x_j}^2}
\newcommand{\pdxixj}{\partial_{x_i x_j}^2}
\newcommand{\pdxjxi}{\partial_{x_j x_i}^2}
\newcommand{\pdmm}{\partial_{m^2}^2}
\newcommand{\pdvi}{\partial_{v_i}}
\newcommand{\pdvj}{\partial_{v_j}}
\newcommand{\pdma}{\partial_{m^\alpha}}
\newcommand{\pdmb}{\partial_{m^\beta}}
\newcommand{\iR}{_{i+\Frac{1}{2}}}
\newcommand{\iL}{_{i-\Frac{1}{2}}}
\newcommand{\iRR}{_{i+1}}
\newcommand{\iLL}{_{i-1}}
\newcommand{\nR}{^{n+\Frac{1}{2}}}
\newcommand{\nRR}{^{n+1}}
\newcommand{\vm}[1]{\left< #1 \right>}
\newcommand{\vmt}[1]{\left< #1 \: |_t\right>}
\newcommand{\vmE}[1]{\left< #1 \: |_E\right>}
\newcommand{\vmL}[1]{\left< #1 \: |_L\right>}
\newcommand{\vmm}[1]{\left< #1 \: |_m\right>}
\newcommand{\vmEs}[1]{\left< #1 \: |_E^*\right>}
\newcommand{\vms}[1]{\left< #1 \: |_v^s \right>}
\newcommand{\pdcg}{\left[\kern-0.12em\left[}
\newcommand{\pdcd}{\right]\kern-0.12em\right]}
\newcommand{\gdcg}{\left[\kern-0.25em\left[}
\newcommand{\gdcd}{\right]\kern-0.25em\right]}
\newcommand{\indic}{{\chi}}

\newtheorem{remarquenum}{Remark}[section]


\makeindex
%%%%%%%%%%%%%%%%%%%%%%%%%%%%%%%%%%%%%%%%%%%%%%%%%%%%%%%%%%%%%%%%%%%%%%%%
% infos sur le document
%\title{Sujet de th�se} 

\date{\today}
%\author{Ga\"el Po\"ette}

\begin{document}

%ins�rer la page de titre
%\maketitle




%remarque, pour les encadr�s oranges pour mettre les formules, il faut faire begin{block}{ (et peut etre mettre en taille \footnotesize


%\setcounter{page}{3}
%ins�rer le sommaire
\textbf{\large{Answer to reviewers for paper (JMLMC-59453): {\it NUMERICAL SPLITTING SCHEMES AS THE CORNERSTONE FOR MINI-BATCH OPTIMIZATION: ON THE IMPORTANCE OF WELL-BALANCED METHODS
}}} \\


\bigskip
\bigskip
\bigskip

We would like to thank the referees for taking the time to carefully read the article. 
The document has been edited under consideration of the remarks made in
the reviews. The attached documents contain a version of the paper where all changes are highlighted
in blue. In the following paragraphs, we address each of the remarks individually.

\ \\
\ \\
In the following paragraphs are our answers to the reviewers' questions and comments. The reviewers' comments are recalled in \textit{italic} while our answers follow. \\ \ \\ 
\begin{enumerate}
  \item {\bf Empirical Evaluation Limited to Determinisitic Settings}

    {\it While the proposed balanced splitting schemes are empirically compelling, the
paper would benefit from a more formal characterization of their convergence
properties in the stochastic mini-batch setting. Specifically, how does the
introduction of the correction term (e.g., $c_n$ in the Speth-style scheme) interact
with stochastic noise when using truly randomized mini-batches (as opposed to
deterministic cyclic updates)?
In practice, most training routines involve sampling batches with replacement or
random reshuffling. It's unclear whether the balance mechanism remains effective
under such stochasticity, especially in high-dimensional, non-convex landscapes.
A theoretical analysis-or at least empirical investigation-of this point would
strengthen the paper and clarify the robustness of the method under realistic
    training conditions.} \\
  \item {\bf Simplified Empirical Evaluation.}

    {\it While the proposed method is evaluated extensively across several synthetic 1D
benchmarks, these settings-though illustrative-are quite limited in complexity
and dimensionality. Since mini-batch optimization is primarily used in high
dimensional and highly non-convex problems (e.g., neural networks), it would be
important to test whether the proposed balanced schemes scale to such settings.
Can the authors provide empirical evidence (even preliminary) that their methods
remain effective in higher-dimensional problems or small neural networks?
Alternatively, a discussion of computational bottlenecks, potential scalability
issues, or future plans to test on real-world tasks would help clarify the practical
    applicability of the approach. }\\

  \item dimensional and highly non-convex problems (e.g., neural networks), it would be
important to test whether the proposed balanced schemes scale to such settings.
Can the authors provide empirical evidence (even preliminary) that their methods
remain effective in higher-dimensional problems or small neural networks?
Alternatively, a discussion of computational bottlenecks, potential scalability
issues, or future plans to test on real-world tasks would help clarify the practical
applicability of the approach.
Minor Comments
Redundant Figures and Length:
a. Several empirical figures across Sections 2, 3, and 5 are quite repetitive,
especially those showing stationary distributions for all learning rates and
optimizers across multiple similar benchmark types. While they are useful
for exhaustiveness, some of them might be moved to an appendix or
summarized more concisely in the main text.
b. For example, Figures 1-4 and 13-17 contain overlapping insights that could
be summarized in fewer plots with aggregate comparisons or referenced
from supplementary material.
Potential to Improve Summary and Discussion:
a. The paper is quite lengthy, and while it is thorough, condensing some
sections-particularly those reviewing existing optimizers and background-
could improve readability. Streamlining parts of Sections 1.0.0.3 and 1.0.0.4,
for instance, would help sharpen the focus on the core contributions and
make the paper more accessible to the reader. 


\item Other Comments:
  \begin{itemize}
    \item  Typo in multiple places: 'litterature' ->'literature'.
    \item  Clarify early on that the proposed methods allow use of a constant learning rate
without requiring full gradient evaluations, as this is particularly relevant for the
machine learning community.
\item Please ensure the caption text in all figures is clear and that axis labels and legends
are readable, particularly in the 1D plots.
\item The github link attached to the paper appears to be not available. Please fix this.
  \end{itemize}

\item Final thoughts:
The clarity of the operator splitting interpretation and the construction of the benchmarks
are appreciated. This paper could generate interest in both the ML optimization and
numerical analysis communities
   \end{enumerate}
\ \\ \ \\

Thanks again for your remarks and questions, we hope the paper is now ready for publication. 

{
\bibliography{./refs}
}


%%%%%%%%%%%%%%%%%%%%%%%%%%%%%%%%%%%%%%%%%%%%%%%%%%%%%%%%%%%%%%%%%%%%%%%%%%%%%%%%%%%%%%%%%%%%%%%%%%%%%%%%%%%%%%%%%%%%%%%%%%%%%%%%%%%%%%%%%%%%%%%%%%%%%
\end{document}


